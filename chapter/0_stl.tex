\chapterimage{chapter_0.png}
\chapter{STL}

\begin{center}
    \pgfornament[width=0.36\linewidth,color=lsp]{88}
\end{center}

\section{pair}
std::pair 是标准库中定义的一个类模板。用于将两个变量关联在一起,组成一个“对”,而且两个变量的数据类型可以是不同的。
\lstinputlisting[style=cpp]{code/stl/pair.cpp}

\section{vector}
std::vector 是 STL 提供的 内存连续的、可变长度 的数组(亦称列表)数据结构。能够提供线性复杂度的插入和删除,以及常数复杂度的随机访问。
\lstinputlisting[style=cpp]{code/stl/vector.cpp}

\section{array}
std::array 是 STL 提供的 内存连续的、固定长度 的数组数据结构。其本质是对原生数组的直接封装。
\lstinputlisting[style=cpp]{code/stl/array.cpp}

\section{deque}
std::deque 是 STL 提供的 双端队列 数据结构。能够提供线性复杂度的插入和删除,以及常数复杂度的随机访问。
\lstinputlisting[style=cpp]{code/stl/deque.cpp}

\section{list}
std::list 是 STL 提供的 双向链表 数据结构。能够提供线性复杂度的随机访问,以及常数复杂度的插入和删除。

list 的使用方法与 deque 基本相同,但是增删操作和访问的复杂度不同。此处列下特殊的(下列函数list提供了特别的实现以供使用):
\lstinputlisting[style=cpp]{code/stl/list.cpp}

\section{set / unordered\_set}
关于 unordered\_set 的哈希冲突,使用自定义哈希函数可以有效避免构造数据产生的大量哈希冲突,代码实现见本册附录。

set 是关联容器,含有键值类型对象的已排序集,搜索、移除和插入拥有对数复杂度。set 内部通常采用红黑树实现。平衡二叉树的特性使得 set 非常适合处理需要同时兼顾查找、插入与删除的情况。\\
和数学中的集合相似,set 中不会出现值相同的元素。如果需要有相同元素的集合,需要使用 multiset。multiset 的使用方法与 set 的使用方法基本相同。(当然也有unordered\_multiset)
\lstinputlisting[style=cpp]{code/stl/set.cpp}

\section{map / unordered\_map}
关于 unordered\_map 的哈希冲突,使用自定义哈希函数可以有效避免构造数据产生的大量哈希冲突,代码实现见本册附录。

map 是有序键值对容器,它的元素的键是唯一的。搜索、移除和插入操作拥有对数复杂度。map 通常实现为红黑树。multimap则允许有重复的Key值。(当然也有 unordered\_multimap)
\lstinputlisting[style=cpp]{code/stl/map.cpp}

\section{string}
std::string 是在标准库 <string>(注意不是 C 语言中的 <string.h> 库)中提供的一个类,本质上是 std::basic\_string<char> 的别称。

\lstinputlisting[style=cpp]{code/stl/string.cpp}

\section{stack}
STL 栈 (std::stack) 是一种后进先出 (Last In, First Out) 的容器适配器,仅支持查询或删除最后一个加入的元素(栈顶元素),不支持随机访问,且为了保证数据的严格有序性,不支持迭代器。
\lstinputlisting[style=cpp]{code/stl/stack.cpp}

\section{queue}
STL 队列 (std::queue) 是一种先进先出 (First In, First Out) 的容器适配器,仅支持查询或删除第一个加入的元素(队首元素),不支持随机访问,且为了保证数据的严格有序性,不支持迭代器。
\lstinputlisting[style=cpp]{code/stl/queue.cpp}

\section{qriority\_queue}
默认容器为vector, 默认算子为 less, 也就是最大堆
\lstinputlisting[style=cpp]{code/stl/priority_queue.cpp}


\section{bitset}
std::bitset 是标准库中的一个存储 0/1 的大小不可变容器。严格来讲,它并不属于 STL。
\lstinputlisting[style=cpp]{code/stl/bitset.cpp}

\section{algorithm}
\subsection{min / max}

% cmath
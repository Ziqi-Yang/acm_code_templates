\chapterimage{chapter_4.png}
\chapter{基础算法}

\begin{center}
    \pgfornament[width=0.36\linewidth,color=lsp]{88}
\end{center}

\section{排序算法}
\subsection{快速排序}
\lstinputlisting[style=cpp]{code/basic/sort/quick_sort.cpp}

\subsection{归并排序}
\lstinputlisting[style=cpp]{code/basic/sort/merge_sort.cpp}

\ifshowLink
相关题目:
    \begin{enumerate}
        \item AcWing 788. 逆序对的数量: \href{https://www.acwing.com/problem/content/790/}{https://www.acwing.com/problem/content/790/}
    \end{enumerate}
\fi

\section{二分}
\subsection{整数二分}
\lstinputlisting[style=cpp]{code/basic/binary_search/int.cpp}

\subsection{浮点数二分}
\lstinputlisting[style=cpp]{code/basic/binary_search/float.cpp}

\ifshowLink
相关题目:
    \begin{enumerate}
        \item AcWing 790. 数的三次方根: \href{https://www.acwing.com/activity/content/problem/content/824/}{https://www.acwing.com/activity/content/problem/content/824/}
    \end{enumerate}
\fi

\section{高精度}
也就是数比较大,不能放在通常的类型中的时候(像python, java 这类的语言本身就实现了这类性质)
\subsection{加法}
\lstinputlisting[style=cpp]{code/basic/high_eps/add.cpp}
\subsection{减法}
\lstinputlisting[style=cpp]{code/basic/high_eps/sub.cpp}
\subsection{乘法}
高精度乘低精度 如 10000位的乘以一个大小为100的
\lstinputlisting[style=cpp]{code/basic/high_eps/multiply.cpp}
\subsection{除法}
高精度除以低精度 如 10000位的除以一个大小为100的
\lstinputlisting[style=cpp]{code/basic/high_eps/div.cpp}

\section{前缀和}
\lstinputlisting[style=cpp]{code/basic/partial_sum.cpp}

\section{差分}
\lstinputlisting[style=cpp]{code/basic/difference.cpp}

\section{位运算}
oi-wiki: \href{https://oi-wiki.org/math/bit/}{https://oi-wiki.org/math/bit/}
\begin{enumerate}
    \item 异或运算的逆运算是它本身,也就是说两次异或同一个数最后结果不变,即 $a \ xor \  b \ xor \  b = 0$。
    \item 操作二进制的某位
    \lstinputlisting[style=cpp]{code/basic/bits/1.cpp}
    \item 判断两非零数符号是否相同
    \lstinputlisting[style=cpp]{code/basic/bits/2.cpp}
    \item 求n的第k位数字
    \lstinputlisting[style=cpp]{code/basic/bits/3.cpp}
    \item 返回n的最后一位1
    \lstinputlisting[style=cpp]{code/basic/bits/4.cpp}
    \item 模拟集合操作 \\
    一个数的二进制表示可以看作是一个集合(0表示不在集合中,1表示在集合中)。比如集合${1,3,4,8}$,可以表示成$(100011010)_{2}$。而对应的位运算也就可以看作是对集合进行的操作。更多见\textit{oi-wiki}
\end{enumerate}

\section{双指针算法}
\textbf{常见问题分类}:
\begin{enumerate}
    \item 对于一个序列,用两个指针维护一段区间 
    \item 对于两个序列,维护某种次序,比如归并排序中合并两个有序序列的操作  
\end{enumerate}
\lstinputlisting[style=cpp]{code/basic/double_pointer.cpp}

\section{离散化}
一般用于给定围很大,但所需范围较小的题
\lstinputlisting[style=cpp]{code/basic/discretization.cpp}

\ifshowLink
相关题目:
    \begin{enumerate}
        \item AcWing 802. 区间和: \href{https://www.acwing.com/problem/content/804/}{https://www.acwing.com/problem/content/804/}
    \end{enumerate}
\fi

\section{区间合并}
\lstinputlisting[style=cpp]{code/basic/interval_merge.cpp}

\section{KMP}
\lstinputlisting[style=cpp]{code/basic/kmp.cpp}